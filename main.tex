\documentclass[10pt,landscape]{article}
\usepackage{multicol}
\usepackage[landscape]{geometry}
\usepackage[procnames]{listings}
\usepackage[parfill]{parskip}
\usepackage[T1]{fontenc}
\usepackage{lmodern}
\usepackage{graphicx}
\usepackage{adjustbox}
\usepackage[table]{xcolor}
\usepackage{mdframed}

% "define" Scala
\usepackage[T1]{fontenc}
\usepackage{microtype}
\usepackage{xcolor}

\colorlet{codebgcolor}{green!20}
\definecolor{commentcolor}{rgb}{0,0.6,0}
\definecolor{gray}{rgb}{0.5,0.5,0.5}
\definecolor{stringcolor}{rgb}{0.58,0,0.82}
\definecolor{light-gray}{gray}{0.75}
\definecolor{keywordcolor}{rgb}{0,0,1}
\definecolor{procnamecolor}{rgb}{1,0,0}

\sbox0{\small\ttfamily A}
\edef\mybasewidth{\the\wd0 }

\lstdefinelanguage{scala}{
  morekeywords={abstract,case,catch,class,def,%
    do,else,extends,false,final,finally,%
    for,if,implicit,import,match,mixin,%
    new,null,object,override,package,%
    private,protected,requires,return,sealed,%
    super,this,throw,trait,true,try,%
    type,val,var,while,with,yield},
  sensitive=true,
  morecomment=[l]{//},
  morecomment=[n]{/*}{*/},
  morestring=[b]",
  morestring=[b]',
  morestring=[b]"""
}

% Default settings for code listings
\lstset{language=scala,
  backgroundcolor = \color{codebgcolor},
  showstringspaces=false,
  columns=fixed, % basewidth=\mybasewidth,
  basicstyle={\small\ttfamily},
  numbers=none,
  numberstyle=\footnotesize\color{gray},
  % identifierstyle=\color{red},
  keywordstyle=\color{keywordcolor},
  commentstyle=\color{commentcolor},
  stringstyle=\color{stringcolor},
  breakatwhitespace=true,
  procnamekeys={def, val, var, class, trait, object, extends},
  procnamestyle=\ttfamily\color{procnamecolor},
  frame=none,
  rulecolor=\color{light-gray},
  xleftmargin=0mm,
  aboveskip=2pt,
  belowskip=2pt,
}

\lstnewenvironment{scala}
{\lstset{language=scala}}
{}
\lstnewenvironment{cpp}
{\lstset{language=C++}}
{}
\lstnewenvironment{bash}
{\lstset{language=bash}}
{}
\lstnewenvironment{verilog}
{\lstset{language=verilog}}
{}


\pagecolor{white}

% Color Pallete
\colorlet{tablekeyscolor}{green!90!yellow!60}
\colorlet{rowcolor}{green!80!yellow!50}
\colorlet{rowlightcolor}{green!70!yellow!40}

% Remove section numbering
\setcounter{secnumdepth}{0}

\geometry{top=.75cm,left=.75cm,right=.75cm,bottom=.75cm}


\pagestyle{empty}
\setlength{\parskip}{0cm}

\makeatletter
\renewcommand{\section}{\@startsection{section}{1}{0mm}%
                                {-0.5ex plus 4ex}%
                                {-0.01ex plus .01ex}%
                                {\normalfont\large\bfseries}}
\renewcommand{\subsection}{\@startsection{subsection}{2}{0mm}%
                                {-0.5ex plus 4ex}%
                                {-0.01ex plus .01ex}%
                                {\normalfont\normalsize\bfseries}}
\renewcommand{\subsubsection}{\@startsection{subsubsection}{3}{0mm}%
                                {-0.01ex plus 0.01ex}%
                                {-0.01ex plus .01ex}%
                                {\normalfont\small\bfseries}}
\makeatother

\begin{document}
\begin{multicols}{3}

\newcommand*\ruleline[1]{\par\noindent\raisebox{.8ex}{\makebox[\linewidth]{\hrulefill\hspace{1ex}\raisebox{-.8ex}{#1}\hspace{1ex}\hrulefill}}}
\renewcommand{\tabcolsep}{.5mm}

\newcommand*\highlightbox[2]{\noindent\adjustbox{bgcolor=#1,minipage=[t]{\linewidth}}{ #2 }}

\ruleline{\Large{\textbf{Chisel3 Cheat Sheet}}}
\begin{center}
Version 0.5 (beta): \today
\end{center}

\fcolorbox{red}{yellow}{ \parbox{0.95\columnwidth} {
\subsection{Notation In This Document}:
\subsubsection{For Functions and Constructors}: \newline
Arguments given as \texttt{kwd:type} (name and type(s)) \newline
Arguments in brackets (\texttt{[...]}) are optional.
\subsubsection{For Operators}: \newline
\texttt{c}, \texttt{x}, \texttt{y} are Chisel \texttt{Data};
\texttt{n}, \texttt{m} are Scala \texttt{Int} \newline
\texttt{w(x)}, \texttt{w(y)} are the widths of \texttt{x}, \texttt{y} (respectively) \newline
\texttt{minVal(x)}, \texttt{maxVal(x)} are the minimum or \newline
\phantom{x} maximum possible values of \texttt{x}
} }

\section{Basic Chisel Constructs } \hrulefill
\subsection{Chisel Wire Operators}:
\begin{scala}
// Allocate a as wire of type UInt()
val x = Wire(UInt())
x := y	// Connect wire y to wire x
\end{scala}

\subsection{When } executes blocks conditionally by \lstinline|Bool|, \newline
\phantom{x} and is equivalent to Verilog \lstinline|if|
\begin{scala}
when(condition1) {
  // run if condition1 true and skip rest
} .elsewhen(condition2) {
  // run if condition2 true and skip rest
} .otherwise {
  // run if none of the above ran
}
\end{scala}

\subsection{Switch } executes blocks conditionally by data
\begin{scala}
switch(x) {
  is(value1) {
    // run if x === value1
  }
  is(value2) {
    // run if x === value2
  }
}
\end{scala}

\subsection{Enum } generates value literals for enumerations \newline
\lstinline|val s1::s2::| ... \lstinline|::sn::Nil| \newline
\lstinline|    = Enum(nodeType:UInt, n:Int)| \newline
\begin{tabular}{l l l}
& \lstinline|s1|, \lstinline|s2|, ..., \lstinline|sn| & will be created as \lstinline|nodeType| literals \\
& & with distinct values \\
& \lstinline|nodeType| & type of \lstinline|s1|, \lstinline|s2|, ..., \lstinline|sn| \\
& \lstinline|n| & element count \\
\end{tabular}

\subsection{Math Helpers}: \newline
    \begin{tabular}{l l}
        \lstinline|log2Ceil(in:Int): Int| & $log_2(\texttt{in})$ rounded up \\
        \lstinline|log2Floor(in:Int): Int| & $log_2(\texttt{in})$ rounded down \\
        \lstinline|isPow2(in:Int): Boolean| & \lstinline|True| if \lstinline|in| is a power of 2 \\
    \end{tabular}

\columnbreak

\section{Basic Data Types} \hrulefill \\
\highlightbox{blue!30}{
    \subsubsection{Constructors}: \newline
        \begin{tabular*}{\columnwidth}{@{\extracolsep{\fill} } l l}
            \lstinline|Bool()| & type, boolean value \\
            \lstinline|true.B| or \lstinline|false.B| & literal values \\
            \lstinline|UInt(32.W)| & type 32-bit unsigned \\
            \lstinline|UInt()| & type, width inferred \\
            \lstinline|77.U| or \lstinline|"hdead".U| & unsigned literals \\
            \lstinline|1.U(16.W)| & literal with forced width \\
            \lstinline|SInt()| or \lstinline|SInt(64.W)| & like UInt \\
            \lstinline|-3.S|  or \lstinline|"h-44".S| & signed literals \\
            \lstinline|3.S(2.W)| & signed 2-bits wide {\em value -1} \\
        \end{tabular*}
    %
    \subsubsection{Bits, UInt, SInt Casts}: reinterpret cast except for:\newline
        \begin{tabular*}{\columnwidth}{@{\extracolsep{\fill} } l l l}
            \lstinline|UInt -->| \lstinline|SInt| & Zero-extend to SInt & \\
        \end{tabular*}
}



\section{State Elements } \hrulefill \\
\highlightbox{blue!30}{
    %
    \subsection{Registers } retain state until updated \newline
        \lstinline[columns=fixed]|val my_reg = Reg(UInt(32.W))| \newline
        Flavors \newline
        \begin{tabular}{l l l}
            & \lstinline|RegInit(7.U(32.w))| & reg with initial value 7 \\
            & \lstinline|RegNext(next_val)| & update each clock, no init\\
            & \lstinline|RegEnable(next,enable)| & update, with enable gate \\
        \end{tabular}
    %
    \subsubsection{Updating}: assign to latch new value on next clock: \newline
        \lstinline|my_reg := next_val|
    %
    \subsection{Read-Write Memory } provide addressable memories \newline
        \lstinline|val my_mem = Mem(n:Int, out:Data)| \newline
        \begin{tabular}{l l l}
            & \lstinline|out| & memory element type \\
            & \lstinline|n| & memory depth (elements) \\
        \end{tabular}
    %
    \subsubsection{Using}: access elements by indexing: \newline
        \lstinline|val readVal = my_mem(addr:UInt/Int)| \newline
        \phantom{x} for synchronous read: assign output to \lstinline[columns=fixed]|Reg| \newline
        \lstinline|mu_mem(addr:UInt/Int) := y|
}

\section{Modules } \hrulefill
\subsubsection{Defining}: subclass \lstinline|Module| with elements, code:
\begin{scala}
class Accum(width:Int) extends Module {
  val io = IO(new Bundle {
    val in = Input(UInt(width.W))
    val out  = Output(UInt(width.W))
  })
  val sum = Reg(UInt())
  sum := sum + io.in
  io.out := sum
}
\end{scala}
\subsubsection{Usage}: access elements using dot notation: \newline
\phantom{x} (code inside a \lstinline|Module| is always running)
\begin{scala}
val my_module = Module(new Accum(32))
my_module.io.in := some_data
val sum := my_module.io.out
\end{scala}

\columnbreak

\subsubsection{Operators}: \newline \newline
{
\rowcolors{2}{rowcolor}{rowlightcolor}
\begin{tabular*}{\columnwidth}{@{\extracolsep{\fill} } l l l}
\rowcolor{tablekeyscolor}
Chisel & Explanation & Width \\
\lstinline|!x| & Logical NOT & \lstinline|1| \\
\lstinline|x && y| & Logical AND & \lstinline|1| \\
\lstinline|x || y| & Logical OR & \lstinline|1| \\
\hline
\lstinline|x(n)| & Extract bit, \lstinline|0| is LSB & \lstinline|1| \\
\lstinline|x(n, m)| & Extract bitfield & \lstinline|n - m + 1| \\
\lstinline|x << y| & Dynamic left shift & \lstinline|w(x) + maxVal(y)| \\
\lstinline|x >> y| & Dynamic right shift & \lstinline|w(x) - minVal(y)| \\
\lstinline|x << n| & Static left shift & \lstinline|w(x) + n| \\
\lstinline|x >> n| & Static right shift & \lstinline|w(x) - n| \\
\lstinline|Fill(n, x)| & Replicate \lstinline|x|, \lstinline|n| times & \lstinline|n * w(x)| \\
\lstinline|Cat(x, y)| & Concatenate bits & \lstinline|w(x) + w(y)| \\
\lstinline|Mux(c, x, y)| & If \lstinline|c|, then \lstinline|x|; else \lstinline|y| & \lstinline|max(w(x), w(y))| \\
\hline
\lstinline|~x| & Bitwise NOT & \lstinline|w(x)| \\
\lstinline|x & y| & Bitwise AND & \lstinline|max(w(x), w(y))| \\
\lstinline|x | y| & Bitwise OR & \lstinline|max(w(x), w(y))| \\
\lstinline|x ^ y| & Bitwise XOR & \lstinline|max(w(x), w(y))| \\
\hline
\lstinline|x === y| & Equality{\small\textcolor{red}{(triple equals)}} & \lstinline|1| \\
\lstinline|x != y| & Inequality & \lstinline|1| \\
\lstinline|x =/= y| & Inequality & \lstinline|1| \\
\hline
\lstinline|x + y| & Addition & \lstinline|max(w(x),w(y))| \\
\lstinline|x +% y| & Addition & \lstinline|max(w(x),w(y))| \\
\lstinline|x +& y| & Addition & \lstinline|max(w(x),w(y))+1| \\
\lstinline|x - y| & Subtraction & \lstinline|max(w(x),w(y))| \\
\lstinline|x -% y| & Subtraction & \lstinline|max(w(x),w(y))| \\
\lstinline|x -& y| & Subtraction & \lstinline|max(w(x),w(y))+1| \\
\lstinline|x * y| & Multiplication & \lstinline|w(x)+w(y)| \\
\lstinline|x / y| & Division & \lstinline|w(x)| \\
\lstinline|x % y| & Modulus & \lstinline|bits(maxVal(y)-1)| \\
\hline
\lstinline|x > y| & Greater than & \lstinline|1| \\
\lstinline|x >= y| & Greater than or equal & \lstinline|1| \\
\lstinline|x < y| & Less than & \lstinline|1| \\
\lstinline|x <= y| & Less than or equal & \lstinline|1| \\
\hline
\lstinline|x >> y| & Arithmetic right shift & \lstinline|w(x) - minVal(y)| \\
\lstinline|x >> n| & Arithmetic right shift & \lstinline|w(x) - n| \\
\end{tabular*}
}

\subsubsection{UInt bit-reduction methods}: \newline \newline
{\rowcolors{2}{rowcolor}{rowlightcolor}
\begin{tabular*}{\columnwidth}{ l l c}
\rowcolor{tablekeyscolor}
Chisel & Explanation & Width \\
\lstinline|x.andR  | & AND-reduce    & \lstinline|1| \\
\lstinline|x.orR   | & OR-reduce     & \lstinline|1| \\
\lstinline|x.xorR  | & XOR-reduce    & \lstinline|1| \\
\end{tabular*}
}
 \newline
As an example to apply the andR method to an SInt use
x.asUInt.andR
 \newline \newline
\columnbreak

%%%%%%%%%%%%%%%%%%%%%%%%%%%%%%%%%%%%%%%%%%%%%%%%%%%%%%%%%%%%%%%%%%%%%%%%%%%%%%%%%%%%%%%%%%%%%%%%%%%%%%%%%%%%%%%
\section{Hardware Generation } \hrulefill \\
\highlightbox{rowcolor}{
    \subsection{Functions } provide block abstractions for code.  Scala
    functions that instantiate or return Chisel types are code generators.
    \subsection{Also}: Scala's \lstinline|if| and \lstinline|for| can be used to control hardware generation
    and are equivalent to Verilog \lstinline|generate if|/\lstinline|for|
    \begin{scala}^^J
        val number=Reg(if(can_be_negative) SInt()^^J
                       else UInt())^^J
    \end{scala}
    will create a Register of type SInt or UInt depending on the value of a \\
    Scala variable
}
%%%%%%%%%%%%%%%%%%%%%%%%%%%%%%%%%%%%%%%%%%%%%%%%%%%%%%%%%%%%%%%%%%%%%%%%%%%%%%%%%%%%%%%%%%%%%%%%%%%%%%%%%%%%%%%
\section{Aggregate Types } \hrulefill \\
\highlightbox{rowlightcolor}{
    \subsection{Bundle } contains \lstinline|Data| types indexed by name
        \subsubsection{Defining}: subclass \lstinline|Bundle|, define components:
        \begin{scala}^^J
            class MyBundle extends Bundle \{^^J
            \ \ val a = Bool()^^J
            \ \ val b = UInt(32.W)^^J
            \}^^J
        \end{scala}
        \subsubsection{Constructor}: instantiate \lstinline|Bundle| subclass:\\
            \lstinline|val my_bundle = new MyBundle()|
        \subsubsection{Inline defining}: define a \lstinline|Bundle| type:
        \begin{scala}^^J
            val my_bundle = new Bundle \{ ^^J
            \ \ val a = Bool()^^J
            \ \ val b = UInt(32.W)^^J
            \} ^^J
        \end{scala}
        \subsubsection{Using}: access elements through dot notation:
        \begin{scala}^^J
            val bundleVal = my_bundle.a^^J
            my_bundle.a := Bool(true)^^J
        \end{scala}
}
\newline \newline
\highlightbox{rowcolor}{
    \subsection{Vec } is an indexable vector of \lstinline|Data| types \\
        \lstinline|val myVec = Vec(elts:Iterable[Data])|
        \begin{tabular}{l l l}
            & \lstinline|elts| & initial element \lstinline|Data| (vector depth inferred)
        \end{tabular}
        \lstinline|val myVec = Vec.fill(n:Int) {gen:Data}|
        \begin{tabular}{l l l}
            & \lstinline|n| & vector depth (elements) \\
            & \lstinline|gen| & initial element \lstinline|Data|, called once per element
        \end{tabular}
        \subsubsection{Using}: access elements by dynamic or static indexing: \\
            \lstinline|readVal := myVec(ind:Data/idx:Int)|
            \lstinline|myVec(ind:Data/idx:Int) := writeVal|
        \subsubsection{Functions}: (\lstinline|T| is the \lstinline|Vec| element's type) \newline
            \begin{tabular}{l l l}
                & \lstinline|.forall(p:T=>Bool): Bool| & AND-reduce \lstinline|p| on all elts \\
                & \lstinline|.exists(p:T=>Bool): Bool| & OR-reduce \lstinline|p| on all elts \\
                & \lstinline|.contains(x:T): Bool| & \lstinline|True| if this contains \lstinline|x| \\
                & \lstinline|.count(p:T=>Bool): UInt| & count elts where \lstinline|p| is \lstinline|True| \\
            \end{tabular}
            \begin{tabular}{l l l}
                & \lstinline|.indexWhere(p:T=>Bool): UInt| & \\
                & \lstinline|.lastIndexWhere(p:T=>Bool): UInt| & \\
                & \lstinline|.onlyIndexWhere(p:T=>Bool): UInt| & \\
            \end{tabular}
}

%%%%%%%%%%%%%%%%%%%%%%%%%%%%%%%%%%%%%%%%%%%%%%%%%%%%%%%%%%%%%%%%%%%%%%%%%%%%%%%%%%%%%%%%%%%%%%%%%%%%%%%%%%%%%%%
\section{Standard Library: Function Blocks } \hrulefill \\
\highlightbox{blue!30}{
    \subsection{Stateless}: \newline
        \lstinline|PopCount(in:Bits/Seq[Bool]): UInt| \newline
        \phantom{x} Returns number of hot (= 1) bits in \lstinline|in|\\
        \lstinline|Reverse(in:UInt): UInt| \newline
        \phantom{x} Reverses the bit order of \lstinline|in|\\
        \lstinline|UIntToOH(in:UInt, [width:Int]): Bits| \newline
        \begin{tabular}{l l l}
            & \multicolumn{2}{l}{Returns the one-hot encoding of \texttt{in}} \\
            & \lstinline|width| & {\em(optional, else inferred)} output width \\
        \end{tabular}
        %
        \lstinline|OHToUInt(in:Bits/Seq[Bool]): UInt| \newline
        \phantom{x} Returns the \lstinline|UInt| representation of one-hot \lstinline|in|\\
        %
        \lstinline|Counter(n:Int]): UInt| \newline
        \phantom{x} .inc() bumps counter returning true when n reached \newline
        \phantom{x} .value returns current value\\
        %
        \lstinline|PriorityEncoder(in:Bits/Iterable[Bool]): UInt| \newline
        \phantom{x} Returns the position the least significant \lstinline|1| in \lstinline|in|\\
        %
        \lstinline|PriorityEncoderOH(in:Bits): UInt| \newline
        \phantom{x} Returns the position of the hot bit in \lstinline|in|\\
        %
        \lstinline|Mux1H(in:Iterable[(Data, Bool]): Data| \newline
        \lstinline|Mux1H(sel:Bits/Iterable[Bool],| \newline
        \lstinline|\ \ \ \ \ \ in:Iterable[Data]): Data| \newline
        \lstinline|PriorityMux(in:Iterable[(Bool, Bits]): Bits| \newline
        \lstinline|PriorityMux(sel:Bits/Iterable[Bool],| \newline
        \lstinline|\ \ \ \ \ \ \ \ \ \ \ \ \ in:Iterable[Bits]): Bits| \newline
        \begin{tabular}{l l l}
            & \multicolumn{2}{l}{A mux tree with either a one-hot select or multiple} \\
            & \multicolumn{2}{l}{\phantom{x} selects (where the first inputs are prioritized)} \\
            & \lstinline|in| & iterable of combined input and select\lstinline|(Bool, Bits)|\\
            & & tuples or just mux input \lstinline|Bits| \\
            & \lstinline|sel| & select signals or bitvector, one per input \\
        \end{tabular}
}
\newline \newline
\highlightbox{blue!30}{
    \subsection{Stateful}: \newline
        \lstinline|LFSR16([increment:Bool]): UInt| \newline
        \begin{tabular}{l l l}
            & \multicolumn{2}{l}{16-bit LFSR (to generate pseudorandom numbers)} \\
            & \lstinline|increment| & {\em(optional, default True)} shift on next clock \\
        \end{tabular}
        \lstinline|ShiftRegister(in:Data, n:Int, [en:Bool]): Data| \newline
        \begin{tabular}{l l l}
            & \multicolumn{2}{l}{Shift register, returns \texttt{n}-cycle delayed input \texttt{in}} \\
            & \lstinline|en| & {\em(optional, default True)} enable \\
        \end{tabular}
}
\newline \newline
%%%%%%%%%%%%%%%%%%%%%%%%%%%%%%%%%%%%%%%%%%%%%%%%%%%%%%%%%%%%%%%%%%%%%%%%%%%%%%%%%%%%%%%%%%%%%%%%%%%%%%%%%%%%%%%
\section{Standard Library: Interfaces } \hrulefill \\
\highlightbox{rowlightcolor}{
    \subsection{DecoupledIO } \mbox{is a \texttt{Bundle} with a ready-valid interface}
        \subsubsection{Constructor}: \newline
            \lstinline|Decoupled(gen:Data)| \newline
            \begin{tabular}{l l l}
                & \lstinline|gen| & Chisel \lstinline|Data| to wrap ready-valid protocol around \\
            \end{tabular}
        \subsubsection{Interface}: \newline
            \begin{tabular}{c c l l}
                & (in) & \lstinline|.ready| & ready \lstinline|Bool| \\
                & (out) & \lstinline|.valid| & valid \lstinline|Bool| \\
                & (out) & \lstinline|.bits| & data \\
            \end{tabular}
}

\highlightbox{rowcolor}{
    \subsection{ValidIO } is a \lstinline|Bundle| with a valid interface
        \subsubsection{Constructor}: \newline
            \lstinline|Valid(gen:Data)| \newline
            \begin{tabular}{l l l}
                & \lstinline|gen| & Chisel \lstinline|Data| to wrap valid protocol around \\
            \end{tabular}
        \subsubsection{Interface}: \newline
            \begin{tabular}{c c l l}
                & (out) & \lstinline|.valid| & valid \lstinline|Bool| \\
                & (out) & \lstinline|.bits| & data \\
            \end{tabular}
}
\newline \newline
\highlightbox{rowlightcolor}{
    \subsection{Queue } is a \lstinline|Module| providing a hardware queue
        \subsubsection{Constructor}: \newline
            \lstinline|Queue(enq:DecoupledIO, entries:Int)| \newline
            \begin{tabular}{l l l}
                & \lstinline|enq| & \lstinline|DecoupledIO| source for the queue \\
                & \lstinline|entries| & size of queue \\
            \end{tabular}
        \subsubsection{Interface}: \newline
            \begin{tabular}{l l l}
                & \lstinline|.io.enq| & \lstinline|DecoupledIO| source (flipped) \\
                & \lstinline|.io.deq| & \lstinline|DecoupledIO| sink \\
                & \lstinline|.io.count| & \lstinline|UInt| count of elements in the queue \\
            \end{tabular}
}
\newline \newline
\highlightbox{rowcolor}{
    \subsection{Pipe }is a \lstinline|Module| delaying input data
        \subsubsection{Constructor}: \newline
            \lstinline|Pipe(enqValid:Bool, enqBits:Data, [latency:Int])| \newline
            \lstinline|Pipe(enq:ValidIO, [latency:Int])| \newline
            \begin{tabular}{l l l}
                & \lstinline|enqValid| & input data, valid component \\
                & \lstinline|enqBits| & input data, data component \\
                & \lstinline|enq| & input data as \lstinline|ValidIO| \\
                & \lstinline|latency| & {\em(optional, default 1)} cycles to delay data by \\
            \end{tabular}
        \subsubsection{Interface}: \newline
            \begin{tabular}{l l l}
                & \lstinline|.io.enq| & \lstinline|ValidIO| source (flipped) \\
                & \lstinline|.io.deq| & \lstinline|ValidIO| sink \\
            \end{tabular}
}
\newline \newline
\highlightbox{rowlightcolor}{
    \subsection{Arbiters } are \lstinline|Module|s connecting multiple producers\newline
        \phantom{x} to one consumer \newline
        \lstinline|Arbiter| prioritizes lower producers \newline
        \lstinline|RRArbiter| runs in round-robin order
        %
        \subsubsection{Constructor}: \newline
            \lstinline|Arbiter(gen:Data, n:Int)| \newline
            \begin{tabular}{l l l}
                & \lstinline|gen| & data type \\
                & \lstinline|n| & number of producers \\
            \end{tabular}
        \subsubsection{Interface}: \newline
            \begin{tabular}{l l l}
                & \lstinline|.io.in| & \lstinline|Vec| of \lstinline|DecoupledIO| inputs (flipped) \\
                & \lstinline|.io.out| & \lstinline|DecoupledIO| output \\
                & \lstinline|.io.chosen| & \lstinline|UInt| input index on \lstinline|.io.out|, \\
                & & does not imply output is valid \\
            \end{tabular}
}

\end{multicols}
\end{document}
